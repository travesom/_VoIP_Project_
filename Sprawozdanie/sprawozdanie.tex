\documentclass[12pt,a4paper]{article}
\title{\Large Politechnika Poznańska \vspace{3cm}\\ \textbf{\LARGE {Telefonia IP}}\\ \vspace{0.7cm} Dokumentacja projektowa \\ \vspace{0.2cm} \textit{NoTitleCall} }
\author{Marcin Złotek 136334 \\ marcin.zlotek@student.put.poznan.pl }
\date{}

\usepackage{geometry}
\geometry{a4paper,left=30mm,right=30mm} %szerokosc marginesów lewy i prawy

\usepackage[T1]{fontenc} 			%\
\usepackage{polski}					% \ustawianie polskich liter
\usepackage[utf8]{inputenc}			% /
\usepackage[english,polish]{babel}	%/
\usepackage{hyperref}


\setlength{\tabcolsep}{18pt} % szerokosc pojedynczej kolumny
\renewcommand{\arraystretch}{1.5} % wyzszy kazdy wiersz  w tabeli
\usepackage{rotating} %obracanie tabel, ...
\usepackage{booktabs} %linie wydzielajace w tabeli
\usepackage{caption} %odstepy
\usepackage{amsmath}
\usepackage{indentfirst} % wciecia w pierwszym akapicie 
\usepackage{longtable} %wielostronicowe tabele
\usepackage{multirow} %komórka ma wiele wierszy
\usepackage{multicol}
\usepackage{enumitem} %zaawansowane listy
\usepackage{graphicx}

\captionsetup[table]{skip=8pt} %przerwa miedzy tytulem tabeli i tabela
\captionsetup[tabular]{belowskip=8pt}


\begin{document}
	\begin{titlepage}
		
		\clearpage
		
		\maketitle
		
		\thispagestyle{empty}
		\vfill
		\begin{center}
			\large Poznań, 2020
		\end{center}
	\end{titlepage}

	\tableofcontents
	\newpage
	
	\section{Ogólna charakterystyka}
	\par Tematem projektu jest system komunikacji głosowej poprzez protokół IP. System pełni rolę skrzynki głosowej, na którą mogą nagrać się użytkownicy. Aplikacja klienta będzie posiadała interfejs tekstowy. \\
	Projekt nosi nazwę \textit{NoTitleCall}, sloganem jest: ,,\textit{listen a lot, talk more}''. % słuchaj dużo, mów więcej / listen a lot, say more
	
	\section{Architektura systemu}
	\par Architektura systemu jest w postaci klient-serwer. Użytkownik będzie się łączył z serwerem w celu nagrania lub odebrania wiadomości.
	\par Do komunikacji z serwerem służy protokół TCP oraz własna implementacja protokołów do nawiązywania połączenia. Do przechowywania danych o użytkownikach serwer wykorzystuje pliki \textit{XML}. Poufność przekazywanych danych jest zapewniona przez protokół \textit{TLS}.
	
	\section{Wymagania}
	
	\subsection{Funkcjonalne}
	\par Wymagania funkcjonalne dla aplikacji klienta i serwera pośredniczącego.
	\subsubsection{Użytkownik niezalogowany/niezarejestrowany}
	\par Wymagania dla aplikacji klienta niezalogowanego/niezarejestrowanego:
	\begin{itemize}
		\item utworzenie konta użytkownika,
		\item utworzenie hasła do konta,
		\item zalogowanie się do systemu.
	\end{itemize}
	\subsubsection{Użytkownik zalogowany}
	\par Wymagania dla aplikacji klienta zalogowanego:
	\begin{itemize}
		\item nagranie wiadomości głosowej,
		\item wybór osoby do której chcemy wysłać wiadomość,
		\item odsłuchanie wiadomości głosowej,
		\item sprawdzenie czy są dostępne nowe wiadomości głosowe.
	\end{itemize}
	\subsubsection{Serwer}
	\par Wymagania dla serwera pośredniczącego w transmisji:
	\begin{itemize}
		\item informowanie użytkownika o błędach w komunikacji,
		\item przechowywanie informacji dotyczącej użytkowników (hasła, pseudonimy),
		\item rejestrowanie nowych użytkowników,
		\item logowanie istniejących użytkowników,
		\item weryfikacja nadawcy danych przy pomocy jednorazowych tokenów,
		\item możliwość zalogowania się administratora systemu,
		\item usuwanie odsłuchanych wiadomości z pamięci.
	\end{itemize}
	
	\subsubsection{Administrator serwera}
	\par Wymagania dla administratora zarządzającego serwerem:
	\begin{itemize}
		\item zalogowanie się na serwer,
		\item sprawdzenia aktualnie zalogowanych użytkowników,
		\item usuwanie kont użytkowników,
		\item sprawdzenie logów z powiadomieniami serwera,
		\item wyłączenia serwera.
	\end{itemize}
	
	\subsection{Pozafunkcjonalne}
	Wymagania pozafunkcjonalne odnoszące się do całego systemu. Są to wymagania dotyczące wydajności, bezpieczeństwa i użyteczności systemu.
	\begin{itemize}
		\item system musi posiadać serwer wielowątkowy,
		\item serwer posiada stały, znany aplikacji klienckiej, adres IP,
		\item hasło służące do zalogowania administratora na serwerze musi być niewidoczne podczas wpisywania,
		\item aplikacja użytkownika posiada tekstowy interfejs użytkownika,
		\item komunikacja klient-serwer jest szyfrowana przy pomocy SSL/TLSv1.2,
		\item system musi działać na systemie operacyjnym Windows 10 lub nowszym,
		\item system powinien przechowywać hasła w postaci skrótu utworzonego funkcją SHA-256,
		\item system wymaga połączenia internetowego o przepustowości 100kB/s (kilobajtów na sekundę) i większej,
		\item nazwą identyfikacyjną użytkownika jest jego adres e-mail,
		\item nazwa identyfikacyjna użytkownika musi być unikalna, 
		\item dane dotyczące użytkowników powinne być przechowywane w plikach XML.
	\end{itemize}
	
	\section{Narzędzia, środowisko, biblioteki}
	\par Zbiór używanych do stworzenia projektu narzędzi i bibliotek. Wymienione zostały także używane środowiska programistyczne (\textit{IDE}), które umożliwiły stworzenie całego systemu. 
	\begin{itemize}
		\item Narzędzia
		\begin{itemize}
			\item C\#,
			\item XML.		
		\end{itemize}
	
		\item Środowisko
		\begin{itemize}
			\item MS Visual Studio 2015 oraz 2019,
			\item RawCap,
			\item Wireshark
			\item MS Visio 2016 
			\item TeXStudio.
		\end{itemize}
	
		\item Biblioteki / Standardy
		\begin{itemize}
			\item X.509,
			\item SSL/TLSv1.2,
			\item NAudio v10.
		\end{itemize}
	\end{itemize}
	
	\newpage
	\section{Diagramy UML}
	
	Diagramy UML przedstawiające budowę i działanie systemu.
	\subsection{Przypadków użycia}
	\par Diagram przypadków użycia z podziałem na aktorów. 
	\begin{figure*}[h!]
		\begin{center}
			\includegraphics*[width=.8\textwidth]{UML_przypadki_uzycia.pdf}
		\end{center}
		\caption{Diagram UML przypadków użycia}
	\end{figure*}
	
	\pagebreak
	\subsection{Stanów}
	\par Diagramy stanów: nagrywanie i odbieranie wiadomości.

	\begin{figure*}[h!]
		\begin{center}
			\includegraphics*[width=.75\textwidth]{UML_stan_rozmowa.pdf}
		\end{center}
		\caption{Diagram UML stanów - nagrywanie wiadomości (u góry) i odbieranie (u dołu)}
	\end{figure*}
	
	\pagebreak
	\subsection{Klas}
	\par Diagramy klas protokołów, aplikacji serwera i klienta.
	\begin{figure*}[h!]
		\begin{center}
			\includegraphics*[width=.8\textwidth]{UML_klas_server.pdf}
		\end{center}
		\caption{Diagram UML klas}
	\end{figure*}
	
	\pagebreak
	\subsection{Sekwencji}
	\par Diagramy sekwencji: próby nawiązania połączenia oraz logowania od systemu.
	
	\begin{figure*}[h!]
		\begin{center}
			\includegraphics*[width=.95\textwidth]{UML_sekwencji_logowanie.pdf}
		\end{center}
		\caption{Diagram UML sekwencji - logowanie}
	\end{figure*}
	
	\pagebreak
	\begin{figure*}[h!]
		\begin{center}
			\includegraphics*[width=.95\textwidth]{UML_sekwencji_nawiazanie_rozmowy.pdf}
		\end{center}
		\caption{Diagram UML sekwencji - przesłanie wiadomości głosowej}
	\end{figure*}
	
	\pagebreak
	\section{GUI}
	\par Projekt interfejsu aplikacji klienta i serwera. 
	
	\begin{figure*}[h!]
		\begin{center}
			\includegraphics*[width=.9\textwidth]{main_screen.png}
		\end{center}
		\caption{Ekran główny - aplikacja klienta}
	\end{figure*}
	\pagebreak
	
	\par Poniżej umieszczono zrzuty ekranu konsoli serwera. Przedstawiono przykładowe komendy dostępne dla administratora po zalogowaniu się. Hasło logowania nie pojawia się przy wpisywaniu, jest to zabieg celowy.
	\begin{figure*}[h!]
		\begin{center}
			\includegraphics*[width=.9\textwidth]{serwer_1.png}
		\end{center}
		\caption{Konsola serwera}
	\end{figure*}
	
	\pagebreak
	\begin{figure*}[h!]
		\begin{center}
			\includegraphics*[width=.8\textwidth]{serwer_2.png}
		\end{center}
		\caption{Konsola serwera}
	\end{figure*}
	
	\begin{figure*}[h!]
		\begin{center}
			\includegraphics*[width=.8\textwidth]{serwer_3.png}
		\end{center}
		\caption{Konsola serwera}
	\end{figure*}
	
	
	
	\pagebreak
	\section{Testy i przebiegi}
	\par Poniżej przedstawiono fragmenty transmisji klient-serwer. Całość przygotowanej transmisji znajduje się w pliku
	\textit{tip\_spr.pcap}. Transmisja (wysłanie i odebranie) przykładowej wiadomości głosowej znajduje się w pliku \textit{tip\_voice.pcap}.
	
	\begin{figure*}[h!]
		\begin{center}
			\includegraphics*[width=.75\textwidth]{testy_1.png}
		\end{center}
		\caption{Przebieg w programie testującym serwer}
	\end{figure*}

	\begin{figure*}[h!]
		\begin{center}
			\includegraphics*[width=.75\textwidth]{testy_2.png}
		\end{center}
		\caption{Przebieg w programie testującym serwer}
	\end{figure*}
	
	\pagebreak
	
	\begin{figure*}[h!]
		\begin{center}
			\includegraphics*[width=.95\textwidth]{wire_1.png}
		\end{center}
		\caption{Fragment przebiegu w programie Wireshark}
	\end{figure*}

	\pagebreak
	\section{Podsumowanie}
	\par Udało się zrealizować projekt w kształcie opisanym w rozdziale 1. System umożliwia nagranie wiadomości głosowej dla innej osoby, do późniejszego odtworzenia.
		
	\subsection{Cele zrealizowane}
	\par Udało się zrealizować własne protokoły realizujące zadania obsługi użytkowników. Aplikacja klienta umożliwia nagrywanie i odsłuchanie wiadomości głosowych. 
	
	\subsection{Problemy}
	Podczas tworzenia aplikacji napotkałem na trudności:
	\begin{itemize}
		\item problem z implementacją szyfrowanej komunikacji z serwerem przy pomocy protokołu SSL,
		\item problemy z działaniem biblioteki NAudio.
	\end{itemize}
	
<<<<<<< HEAD
=======
	\subsection{Perspektywy rozwoju}
	\par Przede wszystkim dokończenie projektu. Następnie perspektywy rozwoju, które zwiększą funkcjonalność systemu:
	\begin{itemize}
		\item rozszerzenie funkcjonalności rejestracji/logowania o możliwość odzyskiwania zapomnianego hasła,
		\item rozszerzenie funkcjonalności administratora po zalogowaniu.
	\end{itemize}
	
	
>>>>>>> parent of a51ba75... update sprawozdanie.pdf
\end{document}