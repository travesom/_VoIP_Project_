\documentclass[12pt,a4paper]{article}
\title{\Large Politechnika Poznańska \vspace{3cm}\\ \textbf{\LARGE {Telefonia IP}}\\ \vspace{0.7cm} Dokumentacja projektowa \\ \vspace{0.2cm} \textit{NoTitleCall} }
\author{Juliusz Horowski 136247 \\ juliusz.horowski@student.put.poznan.pl \and Marcin Złotek 136334 \\ marcin.zlotek@student.put.poznan.pl }
\date{}

\usepackage{geometry}
\geometry{a4paper,left=30mm,right=30mm} %szerokosc marginesów lewy i prawy

\usepackage[T1]{fontenc} 			%\
\usepackage{polski}					% \ustawianie polskich liter
\usepackage[utf8]{inputenc}			% /
\usepackage[english,polish]{babel}	%/
\usepackage{hyperref}


\setlength{\tabcolsep}{18pt} % szerokosc pojedynczej kolumny
\renewcommand{\arraystretch}{1.5} % wyzszy kazdy wiersz  w tabeli
\usepackage{rotating} %obracanie tabel, ...
\usepackage{booktabs} %linie wydzielajace w tabeli
\usepackage{caption} %odstepy
\usepackage{amsmath}
\usepackage{indentfirst} % wciecia w pierwszym akapicie 
\usepackage{longtable} %wielostronicowe tabele
\usepackage{multirow} %komórka ma wiele wierszy
\usepackage{multicol}
\usepackage{enumitem} %zaawansowane listy
\usepackage{graphicx}

\captionsetup[table]{skip=8pt} %przerwa miedzy tytulem tabeli i tabela
\captionsetup[tabular]{belowskip=8pt}


\begin{document}
	\begin{titlepage}
		
		\clearpage
		
		\maketitle
		
		\thispagestyle{empty}
		\vfill
		\begin{center}
			\large Poznań, 2020
		\end{center}
	\end{titlepage}

	\tableofcontents
	\newpage
	
	\section{Ogólna charakterystyka}
	\par Tematem projektu jest system komunikacji głosowej poprzez protokół IP, przeznaczona dla rozmów 1:1. System będzie umożliwiał bezpieczne prowadzenie rozmów, a także czatu tekstowego. Aplikacja klienta będzie posiadała interfejs graficzny. \\
	Projekt nosi nazwę \textit{NoTitleCall}, sloganem jest: ,,\textit{listen a lot, talk more}''. % słuchaj dużo, mów więcej / listen a lot, say more
	
	\section{Architektura systemu}
	\par Architektura systemu jest w postaci klient-serwer. Użytkownik będzie się łączył z serwerem w celu nawiązania połączenia z innym użytkownikiem. Cała transmisja będzie prowadzona poprzez serwer; zarówno rozmowa, czat tekstowy jak i pakiety informacyjne dla serwera. 
	\par Do komunikacji z serwerem służy protokół TCP oraz własna implementacja protokołów do nawiązywania połączenia. Do przechowywania danych o użytkownikach serwer wykorzystuje pliki \textit{XML}. Poufność przekazywanych danych jest zapewniona przez protokół \textit{TLS}.
	
	\section{Wymagania}
	
	\subsection{Funkcjonalne}
	\par Wymagania funkcjonalne dla aplikacji klienta i serwera pośredniczącego.
	\subsubsection{Użytkownik niezalogowany/niezarejestrowany}
	\par Wymagania dla aplikacji klienta niezalogowanego/niezarejestrowanego:
	\begin{itemize}
		\item utworzenie konta użytkownika,
		\item utworzenie hasła do konta,
		\item wybór pseudonimu użytkownika,
	\end{itemize}
	\subsubsection{Użytkownik zalogowany}
	\par Wymagania dla aplikacji klienta zalogowanego:
	\begin{itemize}
		\item prowadzenie czatu głosowego,
		\item prowadzenie czatu tekstowego (z tym samym użytkownikiem co czat głosowy),
		\item wybór osoby z którą chcemy nawiązać kontakt,
		\item nawiązanie połączenia z innym użytkownikiem,
		\item zawieszenie połączenia,
		\item zakończenie połączenia,
		\item wybór motywu graficznego,
		\item zmiana głośności rozmówcy,
		\item wyciszenie rozmówcy,
		\item przechowywanie historię połączeń w formie listy ostatnich 20 połączeń,
		\item wyświetlanie listy kontaktów, w tym przypiętych na samej górze,
		\item opisywanie kontaktów użytkownika (np. zmiana nicku).
	\end{itemize}
	\subsubsection{Serwer}
	\par Wymagania dla serwera pośredniczącego w transmisji:
	\begin{itemize}
		\item informowanie o dostępności użytkownika (dostępny, niedostępny, nie przeszkadzać),
		\item informowanie użytkownika o błędach w komunikacji,
		\item umożlwienie nawiązania połączenia, 
		\item utrzymanie połączenia, 
		\item przechowywanie listy kontaktów, 
		\item przechowywanie informacji dotyczącej użytkowników (hasła, pseudonimy),
		\item rejestrowanie nowych użytkowników,
		\item logowanie istniejących użytkowników,
		\item weryfikacja nadawcy danych przy pomocy jednorazowych tokenów,
		\item możliwość zalogowania się administratora systemu.
	\end{itemize}
	
	\subsubsection{Administrator serwera}
	\par Wymagania dla administratora zarządzającego serwerem:
	\begin{itemize}
		\item zalogowanie się na serwer,
		\item sprawdzenia aktualnie zalogowanych użytkowników,
		\item sprawdzenie logów z powiadomieniami serwera,
		\item wyłączenia serwera.
	\end{itemize}
	
	\subsection{Pozafunkcjonalne}
	Wymagania pozafunkcjonalne odnoszące się do całego systemu. Są to wymagania dotyczące wydajności, bezpieczeństwa i użyteczności systemu.
	\begin{itemize}
		\item system musi posiadać serwer wielowątkowy,
		\item serwer posiada stały, znany aplikacji klienckiej, adres IP,
		\item aplikacja użytkownika posiada graficzny interfejs użytkownika,
		\item aplikacja użytkownika powinna umożliwić wybór jednego z motywów graficznych: ciemny, jasny,
		\item komunikacja klient-serwer jest szyfrowana przy pomocy SSL/TLSv1.2,
		\item komunikacja głosowa pomiędzy użytkownikami musi być szyfrowana,
		\item komunikacja tekstowa pomiedzy użytkownikami musi być szyfrowana,
		\item komunikacja odbywa się 1:1,
		\item czas przesyłu informacji pomiędzy użytkownikami nie powiniem być dłuższy niż 2 sekundy,
		\item system musi działać na systemie operacyjnym Windows 10 lub nowszym,
		\item system powinien przechowywać hasła w postaci skrótu utworzonego funkcją SHA-256,
		\item system powiniem obliczać czas połączenia z dokładnością do 1 sekundy,
		\item system wymaga połączenia internetowego o przepustowości 100kB/s (kilobajtów na sekundę) i większej,
		\item system nie powinien retransmitować danych dźwiękowych,
		\item nazwą identifykacyjną użytkownika jest jego adres e-mail,
		\item nazwa identyfikacyjna użytkownika musi być unikalna, 
		\item dane dotyczące użytkowników powinne być przechowywane w plikach XML.
	\end{itemize}
	
	\section{Narzędzia, środowisko, biblioteki}
	\par Zbiór używanych do stworzenia projektu narzędzi i biliotek. Wymienione zostały także używane środowiska programistyczne (\textit{IDE}), które umożliwiły stworzenie całego systemu. 
	\begin{itemize}
		\item Narzędzia
		\begin{itemize}
			\item C\#,
			\item XML,
			\item LINQ.
		\end{itemize}
	
		\item \'Srodowisko
		\begin{itemize}
			\item MS Visual Studio,
			\item RawCap,
			\item Wireshark
			\item MS Visio 2016 
			\item TeXStudio.
		\end{itemize}
	
		\item Biblioteki / Standardy
		\begin{itemize}
			\item X.509,
			\item SSL/TLSv1.2.
		\end{itemize}
	\end{itemize}
	
	\newpage
	\section{Diagramy UML}
	
	Diagramy UML przedstawiające budowę i działanie systemu.
	\subsection{Przypadków użycia}
	\par Diagram przypadków użycia z podziałem na aktorów. 
	\begin{figure*}[h!]
		\begin{center}
			\includegraphics*[width=.8\textwidth]{UML_przypadki_uzycia.pdf}
		\end{center}
		\caption{Diagram UML przypadków użycia}
	\end{figure*}
	
	\pagebreak
	\subsection{Stanów}
	\par Diagramy stanów: stanu użytkownika zalogowanego i rozmowy pomiędzy użytkownikami.
	\begin{figure*}[h!]
		\begin{center}
			\includegraphics*[width=.75\textwidth]{UML_stan_zalogowany.pdf}
		\end{center}
		\caption{Diagram UML stanów - użytkownik zalogowany}
	\end{figure*}

	\begin{figure*}[h!]
		\begin{center}
			\includegraphics*[width=.65\textwidth]{UML_stan_rozmowa.pdf}
		\end{center}
		\caption{Diagram UML stanów - rozmowa użytkownika z innym}
	\end{figure*}
	
	\pagebreak
	\subsection{Klas}
	\par Diagramy klas protokołów, aplikacji serwera i klienta.
	\begin{figure*}[h!]
		\begin{center}
			\includegraphics*[width=.8\textwidth]{UML_klas_server.pdf}
		\end{center}
		\caption{Diagram UML klas - serwer}
	\end{figure*}
	
	\pagebreak
	\subsection{Sekwencji}
	\par Diagramy sekwencji: próby nawiązania połączenia oraz logowania od systemu.
	
	\begin{figure*}[h!]
		\begin{center}
			\includegraphics*[width=.95\textwidth]{UML_sekwencji_logowanie.pdf}
		\end{center}
		\caption{Diagram UML sekwencji - logowanie}
	\end{figure*}
	
	\pagebreak
	\begin{figure*}[h!]
		\begin{center}
			\includegraphics*[width=.95\textwidth]{UML_sekwencji_nawiazanie_rozmowy.pdf}
		\end{center}
		\caption{Diagram UML sekwencji - nawiązanie połączenia}
	\end{figure*}
	
	\pagebreak
	\section{GUI}
	\par Projekt interfejsu graficznego. 
	
	\begin{figure*}[h!]
		\begin{center}
			\includegraphics*[width=.7\textwidth]{login_screen.png}
		\end{center}
		\caption{Ekran logowania}
	\end{figure*}
	
	\begin{figure*}[h!]
		\begin{center}
			\includegraphics*[width=.7\textwidth]{main_screen.png}
		\end{center}
		\caption{Ekran główny}
	\end{figure*}
	
	\pagebreak
	\section{Testy i przebiegi}
	\par Poniżej przedstawiono fragmenty transmisji klient-serwer. Całość przygotowanej transmisji znajduje się w pliku
	\textit{tip\_spr.pcap}. 
	
	\begin{figure*}[h!]
		\begin{center}
			\includegraphics*[width=.75\textwidth]{testy_1.png}
		\end{center}
		\caption{Przebieg w programie testującym serwer}
	\end{figure*}

	\begin{figure*}[h!]
		\begin{center}
			\includegraphics*[width=.75\textwidth]{testy_2.png}
		\end{center}
		\caption{Przebieg w programie testującym serwer}
	\end{figure*}
	
	\pagebreak
	
	\begin{figure*}[h!]
		\begin{center}
			\includegraphics*[width=.95\textwidth]{wire_1.png}
		\end{center}
		\caption{Fragment przebiegu w programie Wireshark}
	\end{figure*}

	
	
	\pagebreak
	\section{Podsumowanie}
	\par Projektu nie udało się dokończyć z powodu braku zaangażowania ze strony współautora pracy. Ukończono prace nad serwerem, protokołami komunikacyjnymi, szyfrowanie połaczeń. Brakuje aplikacji klienta i obsługi dźwięku.
			
	\section{Podział pracy}
	
	\begin{table*}[h!]
		\begin{center}
			\label{tab:table1}
			\begin{tabular}{|p{.3\textwidth}|p{.6\textwidth}|}
				
				\hline
				\multirow{3}{*}{\large\textbf{Juliusz Horowski}} & aplikacja klienta (w tym GUI) - nie zrobiono\\
				\cline{2-2} & obsługa dźwięku - nie zrobiono \\
				\cline{2-2} & sprawozdanie końcowe \\
				\hline
				\multirow{5}{*}{\large\textbf{Marcin Złotek}} & aplikacja serwera \\
				\cline{2-2} & protokoły komunikacyjne \\
				\cline{2-2} & mechanizm logowania i rejestracji \\
				\cline{2-2} & szyfrowanie komunikacji klient-serwer \\
				\cline{2-2} & sprawozdanie końcowe \\
				\hline

			\end{tabular}
		\end{center}
	\end{table*}
	
	\subsection{Cele zrealizowane}
	\par Udało się zrealizować własne protokoły realizujące zadania obsługi użytkowników i połączeń. Jeśli chodzi o aplikacje klienta to zostało ukończone częściowo GUI oraz teoretyczne założenia dotyczące działania. Chodzi m.in. o diagramy UML komunikacji, klas itp. 
	
	\subsection{Cele niezrealizowane}
	\par Nie udało się ukończyć projektu ze powodu braku współpracy ze współautorem projektu. 
	
	\subsection{Problemy}
	Podczas tworzenia aplikacji napotkaliśmy na trudności:
	\begin{itemize}
		\item problem z implementacją szyfrowanej komunikacji z serwerem przy pomocy protokołu SSL,
		\item brak współpracy 
	\end{itemize}
	
	\subsection{Perspektywy rozwoju}
	\par Przede wszystkim dokończenie projektu. Następnie perspektywy rozwoju, które zwiększą funkcjonalność systemu:
	\begin{itemize}
		\item rozszerzenie funkcjonalności rejestracji/logowania o możliwość odzyskiwania zapomnianego hasła,
		\item rozszerzenie funkcjonalności administratora po zalogowaniu,
	\end{itemize}
	
	
\end{document}